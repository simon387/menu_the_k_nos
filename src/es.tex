\documentclass[12pt]{article}
\usepackage[spanish]{babel}
\usepackage[utf8]{inputenc}
\usepackage[T1]{fontenc}
\usepackage[textwidth=12cm,centering]{geometry}
\usepackage[x11names]{xcolor}
\usepackage{background}
\usepackage{tikz}

\newcommand*\wb[3]{%
        {\fontsize{#1}{#2}\usefont{U}{webo}{xl}{n}#3}}

% The page frame
\SetBgColor{Goldenrod3}
\SetBgAngle{0}
\SetBgScale{1}
\SetBgOpacity{1}
\SetBgContents{%
    \begin{tikzpicture}
        \node at (0.5\paperwidth,0) {\wb{80}{34}{E}\rule[60pt]{.2\textwidth}{0.4pt}%
        \raisebox{55pt}{%
            \makebox[.6\textwidth]{\ \fontsize{24}{29}\selectfont\scshape Carta Tés K-nos }}%
        \rule[60pt]{.2\textwidth}{0.4pt}\wb{80}{34}{F}};
        \node at (2,-0.5\textheight) {\rule{0.4pt}{.8\textheight}};
        \node at (19.5,-0.5\textheight) {\rule{0.4pt}{.8\textheight}};
        \node at (0.5\paperwidth,-\textheight) {\wb{80}{34}{G}\rule[-10pt]{\textwidth}{0.4pt}\wb{80}{34}{H}} ;
    \end{tikzpicture}%
}

% colorize text
\newcommand*\ColText[1]{\textcolor{Goldenrod3}{#1}}

% a tabular* for each food group
\newenvironment{Group}[1]
{\noindent\begin{tabular*}{\textwidth}{@{}p{1\linewidth}@{\extracolsep{\fill}}r@{}}
{\fontsize{24}{29}\selectfont\ColText{#1}}
              \\[0.8em]}
              {
\end{tabular*}}

% to format each entry
\newcommand*\Entry[1]{%
    \sffamily#1}

% to format each subentry
\newcommand*\Expl[1]{%
    \hspace*{1em}\footnotesize #1}

\pagestyle{empty}

\begin{document}

    \begin{Group}{FRUTAS, HIERBA Y ROOIBOS}
        \Entry{TÉ MANGO} \\
        \Expl{procedente de la india, produce un save, aromatico y de sabor pieno, con algo de fruto maduro} \\
        \Entry{PIÑA COLADA} \\
        \Expl{Coco, piña y ron, inspirado en el tropical refresco. En verano resulta delicios bien frío. Combina bien solo y con hielo tras una infusión de 4 minutos a 95°} \\
        \Entry{SUEÑOS TROPICALES} \\
        \Expl{Mezcla escaramujo, pasas, hibisco y aroma natural de jackfurit (fruta de jack). Es Recomendable tomar solo tras una infusión de 5 minutos a 95°} \\
        \Entry{MANZANILLA DULCE FLOR} \\
        \Expl{La manzanilla es una planta muy conocida y utilizada desde antiguo. Los egipcios, griecos y romanos ya la utilizaban contra las enfermedades del hígado y los dolores intestinales. Es recomendable tomar solo tras una infusión de 5 minutos a 95°.} \\
        \Entry{POLEO MENTA} \\
        \Expl{Es una de las especies más cnocidas del género Mentha. En infusión es un excelente tónico estomacal y colabora para hacer la digestión. Estimulante general contra el resfriado. Es recomendable tomar solo tras una infusión de 5 minutos a 95°} \\
        \Entry{TILA} \\
        \Expl{La tila es la flor del tilo. El estrés y la ansiedad producidos por el estilo de vida actual, hacen de estas flores uno de los remedios naturales más utilizados. Es recomendable tomar solo tras una infusión de 5 minutos a 95°} \\
        \Entry{ROOIBOS ORIGINAL} \\
        \Expl{Relajante y antioxidante, esta infusión originaria de Sudáfrica, además de tener un delicado sabor, es muy rica en minerales esenciales: calcio, hierro, flúor, potasio y magnesio. Es recomendable tomar solo tras una infusión de 5 minutos a 95°} \\
        \Entry{ROOIBOS NARANJA} \\
        \Expl{Esta infusión originaria de Sudáfrica, además de tener un delicado sabor, es muy rica en minerales esenciales. Mezcla original que combina con el toque cítrico de la naranja. Es recomendable tomar solo tras una infusión de 5 minutos a 95°}
    \end{Group}
    \vfill
    \begin{Group}{TÉ NEGRO Y TÉ BLANCO}
        \Entry{TÉ NEGRO CHAI} \\
        \Expl{Té negro, ciau, canela, caramomo, vanilla, jengibre, naranja y pimienta rosa; recomendamos tomar con leche} \\
        \Entry{TÉ NEGRO A LA CANELA} \\
        \Expl{Uno de los aromatizados más populares, el Té Negro con canela en rama. Se puede tomar con leche tras una infusión de 5 minutos a unos 95°} \\
        \Entry{TÉ CEILÁN} \\
        \Expl{Característico Té Negro de hoja entera y alargada. Se trata de un "high grown tea" (crecimiento de altura). Fuerte y refrescante, ideal por las mañanas. Combina bien con leche o limón tras una infusión de 4 minutos a unos 95°} \\
        \Entry{TÉ DARJEELING} \\
        \Expl{El champagne de los Tés. Té fermentado de hoja grande de color marrón oscuro, de la región del mismo nombre, en India. Su infusión es ligera y delicada con un característico sabor a moscatel y un pronunciado aroma. Combina bien con leche tras una infusión de 3 minutos a unos 95°. } \\
    \end{Group}


\end{document}